\documentclass[a4paper,man,natbib]{apa6}

\usepackage[english]{babel}
\usepackage[utf8x]{inputenc}
\usepackage{amsmath}
\usepackage{graphicx}
\usepackage[colorinlistoftodos]{todonotes}
\usepackage{natbib}

% In your .tex file
% !TEX program = pdflatex

\title{Mandated Change in Higher Education: a Change Laboratory Perspective}
\shorttitle{PhD Proposal - M Johnstone}
\author{Mark Johnstone}
\affiliation{Faculty of Social Sciences, Department of Social Research \\Lancaster University \\October, 2017}

\abstract{This is my abstract.}

\begin{document}
\maketitle

\section{Introduction}
\todo[inline, color=green!40]{The purpose of an Introduction section is to set out a brief overview of your project... The Introduction will mention the context, your approach, and your research questions, and indicate how the remainder of the document will be structured.}

This project will assess the effectiveness of the Change Laboratory method to support mandated pedagogical change \citep{newcombe1981theory} at an institution of higher learning in Saudi Arabia. The research context is a new, Bachelor of Science degree program at a military training college in Riyadh. This new BSc program is being developed and delivered by a US university. The task is to provide a degree program of the same quality as at the home campus, adapted to the specific needs of the Saudi client and with English as the language of instruction. 

To enhance quality of instruction \citep{masoumi2012quality} and to provide coherence and consistency throughout the program \citep{spelt2015constructively}, project managers have adopted Constructive Alignment (CA) \citep{biggs2011teaching} as the core organizing framework for the project. According Biggs and Tang (2011), ``Constructive alignment is a design for teaching... based on the twin principles of constructivism in learning, and alignment both of teaching and of assessment tasks to the intended learning outcomes,'' (81, 108). 

CA affects several activities traditionally regarded as the responsibility of instructors including curriculum design, teaching practices and assessment strategies. As a ``design for teaching'', Constructive Alignment describes an approach to teaching and learning that may require faculty to adopt new pedagogical roles [ref]. When implemented as institutional policy, Constructive Alignment can disrupt faculty teaching habits, challenge beliefs about teaching and learning, or compel faculty to acquire new skills and new competencies [ref]. This study will assess the effectiveness of the Change Laboratory method as a developmental intervention to moderate pedagogical change within a higher education teaching cohort.

\subsection{Context}
\label{sec:background}

\todo[inline, color=green!40]{The areas of research literature that you are drawing from, and to which you are hoping to contribute.}

Although Constructive Alignment can be implemented at course, program, department, or institutional levels, institutional implementation appears to be more common at lower levels: most studies focus on implementation of the framework in courses \citep{joseph2012using} or at program level  \citep{spelt2015constructively}. In many cases, CA is adopted by individual instructors as a design principle or teaching strategy \citep{masoumi2012quality} and does not reach beyond their own courses.

I propose to observe an example of the deliberate reform of an institution of higher education as it unfolds. I have chosen an intervention research design because I am an active participant in the reform process. The most relevant literature to this relates to design based research, cultural historical activity theory, expansive learning, and the Change Laboratory - an intervention research methodology. I have chosen the Change Laboratory because it has a successful track record in in industry, including technology and health care, but has not as yet been widely applied to Higher Education \citep{bligh2015change}. One of the intended outcomes of this project is to further explore the application of the Change Laboratory methodology to Higher Education.
\newline

\todo[inline, color=green!40]{The Context of policy and changes in policy.}

My research site is a government higher education institution in which reform has been mandated but poorly defined. For some time, local administrators believed that academic programs in place were not effective and, as a result, most were terminated in the early 2000s. The institution turned to training as its primary activity and students in need of degree level qualifications were sent abroad. This policy succeeded to some degree, however, the education students received was remote from local contexts. A decision was made to resume the academic program and a foreign university was contracted to design and deliver it on site.

Notions of skill transfer are common in the region as is clear from the popularity of off-the-shelf and boiler-plate solutions. The common assumption is that a foreign Higher Education provider can address local needs and, over a period of years, transfer its ability to do so to the local institution. 
\newline

\todo[inline, color=green!40]{The practices you will be investigating.}

My intention is to investigate the change process in higher education where change is institutionally mandated but neither directed nor defined in a clear and coherent way. I believe this is a common situation as many stakeholders remain focused on outcomes while overlooking how these might be achieved. Institutional measures of success may not dwell on curricula, materials, practice, or pedagogy but merely look at results in terms of whether or not graduates are prepared for their eventual institutional roles or not. In our case, local faculty may or may not agree that their usual practice is in need of reform and foreign faculty may or may not agree that their usual practice is any different from that of their local counterparts, nevertheless, the first are expected ``to learn'' from the second. 
\newline

\todo[inline, color=green!40]{Your own background and motivation.}

This research project interests me because it relates directly to my professional practice, aligns with my beliefs about equitable and effective work practices, and has the potential to make a lasting and valuable contribution to the organizations and institutions that I work within.

The methodologies that I have chosen are also well aligned to my  personal beliefs and world view. I first encountered intervention research in the work of Ann Brown \citep{brown1992design} and it spoke directly to me as a classroom teacher. Later, in Change Laboratory \citep{Virkkunen2013change}, I saw how this type of research activity might be organized in a systematic way not merely to discover something new but to benefit my research site in a concrete way. 


\section{Research Questions}
\todo[inline, color=green!40]{Formulate as questions. Number them explicitly (RQ1, RQ2 etc. Organize in a hierarchy main to supporting. Minimize specialist termiology - cast as stand alone questions. Ensure that all are answerable. Place toward end of Intro Section.}
\begin{description}
    \item[RQ1]How does research make use of Activity Theory?
    \begin{description}
        \item[RQ1.1]What methodologies are used alongside Activity Theory?
        \item[RQ1.2]What data collection methods are used alongside Activity Theory?
        \item[RQ1.3]Which concepts from within Activity Theory are highlighted within the research analysis as presented?
    \end{description}
\end{description}

\begin{description}
   \item[RQ2] How does research published in Higher Education? 
   \begin{description}
      \item[RQ2.1] First level item
      \item[RQ2.2] Second level item
      \item[RQ2.3] Second level item
   \end{description}
\end{description}
\begin{description}
   \item[RQ3] How does research published in Higher Education? 
   \begin{description}
      \item[RQ3.1] First level item
      \item[RQ3.2] Second level item
      \item[RQ3.3] Second level item
   \end{description}
\end{description}

\newpage

\subsection{Signposting}

\todo[inline, color=green!40]{Indicate to readers what the structure of the rest of the document is like by taking account of top level section headings and explaining their purpose. Repeat at the beginning of each section to map out sub sections, and at the end of sections, making brief reference to the next section.}

\section{Literature Review}
\todo[inline, color=green!40]{Aim of this section is to set out and justify a research plan: locate the project within a wider field of inquiry and explain how this might add to that field.  Focus on empirical literature - theoretical sources go to the Theoretical Framework section. Include brief intro, v brief discussion of proximate topics, identify relevant proximate fields, how you located sources, subsections to each proxmiate field of inquiry - state position and critique. End - identify key gaps and points of emphasis.}

\section{Theoretical Framework}
\todo[inline, color=green!40]{The purpose of ....}

\section{Methodology}

\todo[inline, color=green!40]{The purpose of ....}

\section{Conclusion}

\todo[inline, color=green!40]{The purpose of ....}

\section{Timetable / Plan of Action}

\todo[inline, color=green!40]{The purpose of ....}

\section{Thesis Structure}

\todo[inline, color=green!40]{The purpose of ....}

\section{References}

\todo[inline, color=green!40]{The purpose of ....}

\subsection{Hoard}
(BB)
The Introduction will mention the context, your approach, and your research questions, and indicating how the remainder of the document will be structured. I will say something about each of these matters in turn in what follows. I shall encapsulate each of those topics within its own subsection, though this is not meant to imply that you must reproduce that subsection structure in your Confirmation document unnecessarily.

(BB)
There are a number of different focus points that I am grouping together here under this heading of Context. The relative weight given to each focus point, and the order in which you present them to the reader, will vary between Confirmation Documents. However, I do think that when writing an Introduction section you should consider saying something about: (\textit{Bligh, Notes on Confirmation Documents})

(MCJ)
government institution had prepared newly recruited employees for positions in government service within the institution's own domain of activity for many years. The target domain is dynamic and rapidly evolving and recent developments had given rise to new training requirements. These requirements rendered existing programs remote from perceived institutional needs. In response to these challenges, the Education Unit was directed to better prepare its students for their eventual professional roles within the organization. 



The Education Unit is constituted as a traditional Higher Education institution and is run by professional academic faculty but is subordinate to a cadre of directors who are not specialized in teaching. When the unit continued to produce graduates who were deemed by their eventual supervisors as unprepared, the three-year programs were cancelled and the education unit was directed to organize short in-service training programs only. Those in need of longer programs were sent to Higher Education programs abroad.

As the foreign programs did not fully meet institutional needs either, HE programs are being brought back to the home institution but under the direction of a foreign HE provider. 

************

In a sense, education itself is about change and working systems must be able to adapt to change, even to drive it. Institutional needs often mitigate against change give rise to tensions among competing interests of institutional stability, instructional effectiveness, and relevance. How education changes is a philosophical question: how educational cultures can change themselves once a need has been recognized is a pracical matter that interests me directly both my job and in my practice. 

My own experience in working with others as a classroom teacher, as a supervisor of teachers, and as a program administrator has been that the most effective means of moving forward is to trust others to make decisions about their own practice. I was drawn to intervention research 

Educational reform is important in Saudi Arabia given its increasingly young population, growing cultural and generational divides, and the inevitable need to develop human and social capital in anticipation of an emergent post-oil economy (REF). 


Previous change initiatives have been inadequate and  officials suspect that this is because these initiatives were top-down and institutionally driven. The same officials assert that some stakeholders were actively resistant to change. The Change Laboratory method offers an alternative to the formerly top-down approach and has the potential to overcome resistance by extending agency and validating existing expertise. The question here is whether those stakeholders were resistant change in itself, or whether they were resisting their own disenfranchisement when they perceive the change process as directed from elsewhere. 

%\bibliographystyle{plainnat} % or try abbrvnat or unsrtnat
\bibliography{TEL_Proposal_2015} % refers to example.bib

\end{document}

%
% Please see the package documentation for more information
% on the APA6 document class:
%
% http://www.ctan.org/pkg/apa6
%